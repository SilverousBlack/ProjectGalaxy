\documentclass[lettersize,journal]{IEEEtran}
\usepackage{amsmath,amsfonts}
\usepackage{algorithmic}
\usepackage{algorithm}
\usepackage{array}
\usepackage[caption=false,font=normalsize,labelfont=sf,textfont=sf]{subfig}
\usepackage{textcomp}
\usepackage{stfloats}
\usepackage{url}
\usepackage{verbatim}
\usepackage{graphicx}
\usepackage{cite}
\hyphenation{op-tical net-works semi-conduc-tor IEEE-Xplore}

\begin{document}

\title{ProjectGalaxy: Astrophysics Software Tools for Visualization and More}
\author{
    Engr. Julian Caleb Segundo, CpE \\
    Software Developer / Engineer, Data Scientist \\
    mjcdsegundo@tip.edu.ph  
}
\maketitle

\begin{abstract}
This project is a personal undertaking, open to the public for contribution and use via open source licensing under the MIT License (\url{https://mit-license.org}). This project shall, at all times, remain under same or similarly permissive open source licensing.  This project aims to develop a software that encapsulates a myriad of tools and functionalities including applications, web servable or otherwise. The software is for purposes of usage in Astrophysics, primarily, but not limited to visualization. Furthermore, the software shall be platform independent and operating system agnostic, to the extent of maximization of user reach.
\end{abstract}
\begin{IEEEkeywords}
Software tools, visualization, astrophysics
\end{IEEEkeywords}

\section{Introduction}
\IEEEPARstart{T}{he} field of astrophysics is a vast theoretical field concerning itself with the macrophenomena governing the universe\cite{ref1}. Loosely speaking, this branch of astronomy deals with software for uses in visualization and simulation of theoretical concepts applied to tangible data. In this matter, pursuants of such field may often themselves in the complexities of writing software, in terms of scripts or programs. \\
The applications of researches and theoretics in astrophysics often lead to break through in other fields; such as magnetic resonance in medicine and spectroscopy in chemistry. For in this matter, the field has a undeniable significant impact already to the entire human race which, in the author's humble opinion, should be highly invested in with resources. \\
There being, this project shall, in concept, provide help to those that study the field of astrophysics by providing accessible tools and functionalities that reduce the complixities of dealing with software. Therefore, the project in its principles should promote {\bf{no merchantibility}} with the goal of providing the software as a service to a broad audience, whether those in the academe, professional or public service. 

\section{Intent and \\ Design Motivation}
It is no question that dealing with software can be difficult, especially in programming custom software. Furthermore, complexities of data, with its parsing, handling and processing can be much more difficult. Therefore, it is both well-deserved and proper to contribute to the advancement of humanity with the development of technology that alleviate these difficulty, with the hope uncovering more of the secrets of the universe. \\
With this, the software should be designed with wide usability in mind in its programmable and especially to user-facing components. As a cornerstone to allow more individuals to explore this science, the project should also promote accessibility and readability.

\section{Development Requirements}
The project would require a number of resources to come into fruition, aptly grouped into equipment, manpower and expertise. These resources are and shall be used by developer(s) for the development of the project and its components and requisites.

\subsection{Computational Power and Equipment}

\subsection{Development Manpower}

\subsection{Technical Expertise}

\section{Principal Standards and \\ Guiding Principles}

\section{Licensing}

\section{Proposed Software Requirements}

\section{Roadmap}

\begin{thebibliography}{1}
\bibliographystyle{IEEETran}

\bibitem{ref1}
Dan Maoz. (2016). Astrophysics in a Nutshell (2nd ed.). Princeton University Press. https://books.google.com.ph/books?id=bmBeCwAAQBAJ
    
\end{thebibliography}

\end{document}
